\newcommand*{\IIC}{I$^2$C}

\pageauthor{Schuh}
\section{Allgemeines}
\label{sec:allgemeines}

\subsection{Entstehungsgeschichte}
\label{sec:entstehungsgeschichte}

Seit mehreren Jahren wird in der HTBL-Hollabrunn, ein \gls{ARM} Cortex-M3 \gls{Minimalsystem}, für die Ausbildung unserer Schüler, im Bereich \enquote{embedded Systems} eingesetzt.

Wie schon im Abstract beschrieben geht es bei dem neuen System darum, sich neunen Technologien und Anwenderszenarien zu öffnen beziehungsweise schnelles Prototyping zu ermöglichen. Mit Hilfe des Nextion-Touchscreen-Displays wird ein modernes \gls{MMI} bereitgestellt, um Anwendungen leichter und interaktiv bedienbar zu machen. Das Audio-Interface ermöglicht es, Anwendungen für digitale Signalverarbeitung (z.B. digitale Filter) zu realisieren. Das Arduino-Interface ermöglicht es, verschiedenste Arduino-Shields für den Unterricht einzusetzen. Diese Schnittstellen, sowie die Schnittstellen für WLAN, Bluetooth und Funkmodule ermöglichen es auf schnelle Art und Weise Konzepte für Diplomarbeiten zu evaluieren.

\subsection{Anwendungsszenarien}
\label{sec:anwendungsszenarien}

Das gesamte \gls{ARM}-\gls{Minimalsystem} soll dazu beitragen, mit Hilfe einer Vielzahl an Schnittstellen, hardwarenahe Programmierung zu erlernen, sowie das bauen und testen von Prototypen zu erleichtern. Weiters kann aufgrund, des auf der \gls{Basisplatine} vorhandenen Arduino-Sockels eine Kompatibilität zu allen Arduino-Shields erreicht werden, welche nun über das \gls{Core-Modul} angesteuert werden können.

Das Hauptaugenmerk wurde jedoch auf folgende Anwendungen gelegt:

\begin{itemize}
    \item \gls{DSV}
    \item Kommunikation mit diversen Schnittstellen (\IIC{}, SPI, UART, 1-Wire, \dots{})
    \item Hardwarekompatibilität zu Arduino-Shields
    \item \gls{GUI}
\end{itemize}

\subsection{Systemaufbau}
\label{sec:systemaufbau}

Das neue \gls{ARM}-\gls{Minimalsystem} kann prinzipiell in drei voneinander getrennten Platinen unterteilt werden. Diese Module wären die \gls{Basisplatine}, das \gls{Core-Modul} und der \gls{USB-to-UART} Adapter. Jedes dieser Module erfüllt einen bestimmten Zweck, welcher schlussendlich zum Gesamtsystem beträgt.

\todo[inline]{Gesamtbild der Schaltungen}