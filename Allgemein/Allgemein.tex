\pageauthor{Mieke, Reischl, Schuh}
\section{Allgemeines}
\label{sec:allgemeines}

\subsection{Entstehungsgeschichte}
\label{sec:entstehungsgeschichte}

Seit mehreren Jahren wird in der HTBL-Hollabrunn, ein ARM Cortex-M3 \gls{Minimalsystem}, für die Ausbildung unserer Schüler, im Bereich \enquote{embedded Systems} eingesetzt.

Wie schon im Abstract beschrieben geht es bei dem neuen System darum, sich neunen Technologien und Anwenderszenarien zu öffnen beziehungsweise schnelles Prototyping zu ermöglichen. Mit Hilfe des Nextion-Touchscreen-Displays wird ein modernes \gls{MMI} bereitgestellt, um Anwendungen leichter und interaktiv bedienbar zu machen. Das Audio-Interface ermöglicht es, Anwendungen für digitale Signalverarbeitung (z.B. digitale Filter) zu realisieren. Das Arduino-Interface ermöglicht es, verschiedenste Arduino-Shields für den Unterricht einzusetzen. Diese Schnittstellen, sowie die Schnittstellen für WLAN, Bluetooth und Funkmodule ermöglichen es auf schnelle Art und Weise Konzepte für Diplomarbeiten zu evaluieren.
